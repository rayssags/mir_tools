%% Generated by Sphinx.
\def\sphinxdocclass{report}
\documentclass[letterpaper,10pt,english]{sphinxmanual}
\ifdefined\pdfpxdimen
   \let\sphinxpxdimen\pdfpxdimen\else\newdimen\sphinxpxdimen
\fi \sphinxpxdimen=.75bp\relax
\ifdefined\pdfimageresolution
    \pdfimageresolution= \numexpr \dimexpr1in\relax/\sphinxpxdimen\relax
\fi
%% let collapsible pdf bookmarks panel have high depth per default
\PassOptionsToPackage{bookmarksdepth=5}{hyperref}

\PassOptionsToPackage{warn}{textcomp}
\usepackage[utf8]{inputenc}
\ifdefined\DeclareUnicodeCharacter
% support both utf8 and utf8x syntaxes
  \ifdefined\DeclareUnicodeCharacterAsOptional
    \def\sphinxDUC#1{\DeclareUnicodeCharacter{"#1}}
  \else
    \let\sphinxDUC\DeclareUnicodeCharacter
  \fi
  \sphinxDUC{00A0}{\nobreakspace}
  \sphinxDUC{2500}{\sphinxunichar{2500}}
  \sphinxDUC{2502}{\sphinxunichar{2502}}
  \sphinxDUC{2514}{\sphinxunichar{2514}}
  \sphinxDUC{251C}{\sphinxunichar{251C}}
  \sphinxDUC{2572}{\textbackslash}
\fi
\usepackage{cmap}
\usepackage[T1]{fontenc}
\usepackage{amsmath,amssymb,amstext}
\usepackage{babel}



\usepackage{tgtermes}
\usepackage{tgheros}
\renewcommand{\ttdefault}{txtt}



\usepackage[Bjarne]{fncychap}
\usepackage{sphinx}

\fvset{fontsize=auto}
\usepackage{geometry}


% Include hyperref last.
\usepackage{hyperref}
% Fix anchor placement for figures with captions.
\usepackage{hypcap}% it must be loaded after hyperref.
% Set up styles of URL: it should be placed after hyperref.
\urlstyle{same}

\addto\captionsenglish{\renewcommand{\contentsname}{Contents:}}

\usepackage{sphinxmessages}
\setcounter{tocdepth}{1}



\title{LINFIT}
\date{Oct 11, 2022}
\release{Today}
\author{Ray}
\newcommand{\sphinxlogo}{\vbox{}}
\renewcommand{\releasename}{Release}
\makeindex
\begin{document}

\pagestyle{empty}
\sphinxmaketitle
\pagestyle{plain}
\sphinxtableofcontents
\pagestyle{normal}
\phantomsection\label{\detokenize{index::doc}}


\sphinxstepscope


\chapter{LINFIT}
\label{\detokenize{LINFIT:linfit}}\label{\detokenize{LINFIT:id1}}\label{\detokenize{LINFIT::doc}}
\sphinxAtStartPar
LINFIT

\phantomsection\label{\detokenize{LINFIT:module-mirspec}}\index{module@\spxentry{module}!mirspec@\spxentry{mirspec}}\index{mirspec@\spxentry{mirspec}!module@\spxentry{module}}\index{mirspec (class in mirspec)@\spxentry{mirspec}\spxextra{class in mirspec}}

\begin{fulllineitems}
\phantomsection\label{\detokenize{LINFIT:mirspec.mirspec}}
\pysigstartsignatures
\pysiglinewithargsret{\sphinxbfcode{\sphinxupquote{class\DUrole{w}{  }}}\sphinxcode{\sphinxupquote{mirspec.}}\sphinxbfcode{\sphinxupquote{mirspec}}}{\emph{\DUrole{n}{wavelength}}, \emph{\DUrole{n}{flux}}, \emph{\DUrole{n}{flux\_error}}}{}
\pysigstopsignatures
\sphinxAtStartPar
Create a new mid\sphinxhyphen{}infrared spectrum object
\begin{quote}\begin{description}
\item[{Parameters}] \leavevmode\begin{itemize}
\item {} 
\sphinxAtStartPar
\sphinxstyleliteralstrong{\sphinxupquote{wavelength}} \textendash{} Wavelength in microns  or specified unit

\item {} 
\sphinxAtStartPar
\sphinxstyleliteralstrong{\sphinxupquote{flux}} \textendash{} Flux in Jansky or specified unit

\item {} 
\sphinxAtStartPar
\sphinxstyleliteralstrong{\sphinxupquote{flux\_error}} \textendash{} Flux error in Jansky  or specified unit

\end{itemize}

\item[{Return type}] \leavevmode
\sphinxAtStartPar
Instance variable

\item[{Returns}] \leavevmode
\sphinxAtStartPar
\begin{description}
\item[{\sphinxstylestrong{spectrum} variable with wavelength in microns and}] \leavevmode
\sphinxAtStartPar
flux and flux error in Jansky.

\end{description}


\end{description}\end{quote}
\index{linfit() (mirspec.mirspec method)@\spxentry{linfit()}\spxextra{mirspec.mirspec method}}

\begin{fulllineitems}
\phantomsection\label{\detokenize{LINFIT:mirspec.mirspec.linfit}}
\pysigstartsignatures
\pysiglinewithargsret{\sphinxbfcode{\sphinxupquote{linfit}}}{\emph{\DUrole{n}{sigma}\DUrole{o}{=}\DUrole{default_value}{1.5}}, \emph{\DUrole{n}{l\_min}\DUrole{o}{=}\DUrole{default_value}{5}}, \emph{\DUrole{n}{l\_max}\DUrole{o}{=}\DUrole{default_value}{15}}, \emph{\DUrole{n}{max\_interations}\DUrole{o}{=}\DUrole{default_value}{50}}}{}
\pysigstopsignatures
\sphinxAtStartPar
Executes the fit of the equation
\(\log F_\nu = \text{slope}\times\log\nu + \text{intercept}\)
by clipping the emission and absorption features where
\(\text{residual} > \sigma\times\text{standard deviation}_\text{residual}\)

\sphinxAtStartPar
\sphinxstylestrong{How to use:}

\sphinxAtStartPar
Initiate a spectrum object

\sphinxAtStartPar
\sphinxcode{\sphinxupquote{spec = mirspec(wavelength, flux, flux\_error)}}

\sphinxAtStartPar
run the method

\sphinxAtStartPar
\sphinxcode{\sphinxupquote{spec.linfit()}}

\sphinxAtStartPar
and call any of the resulting instance variables

\sphinxAtStartPar
\sphinxcode{\sphinxupquote{spec.ten\_flux}}
\begin{quote}\begin{description}
\item[{Parameters}] \leavevmode\begin{itemize}
\item {} 
\sphinxAtStartPar
\sphinxstyleliteralstrong{\sphinxupquote{sigma}} \textendash{} Multiplicative number that determines that a point sigma*standard deviation of the residuals
is excluded before next iteration, effectively clipping the spectrum

\item {} 
\sphinxAtStartPar
\sphinxstyleliteralstrong{\sphinxupquote{l\_min}} \textendash{} Minimum wavelength to use

\item {} 
\sphinxAtStartPar
\sphinxstyleliteralstrong{\sphinxupquote{l\_max}} \textendash{} Maximum wavelength to use

\item {} 
\sphinxAtStartPar
\sphinxstyleliteralstrong{\sphinxupquote{max\_interations}} \textendash{} Maximum number of times to iterate the fit

\end{itemize}

\item[{Return type}] \leavevmode
\sphinxAtStartPar
Instance variables

\item[{Returns}] \leavevmode
\sphinxAtStartPar
Results of the fit \sphinxstylestrong{slope}, \sphinxstylestrong{intercept}, the absorption corrected 10.5 microns
flux (\sphinxstylestrong{ten\_flux}) and the arrays containing the data of each iteration for
\sphinxstylestrong{frequencies}, \sphinxstylestrong{fluxes}, \sphinxstylestrong{slopes}, \sphinxstylestrong{intercepts}, \sphinxstylestrong{residuals}, \sphinxstylestrong{residual\_stds}.

\end{description}\end{quote}

\end{fulllineitems}


\end{fulllineitems}



\renewcommand{\indexname}{Python Module Index}
\begin{sphinxtheindex}
\let\bigletter\sphinxstyleindexlettergroup
\bigletter{m}
\item\relax\sphinxstyleindexentry{mirspec}\sphinxstyleindexpageref{LINFIT:\detokenize{module-mirspec}}
\end{sphinxtheindex}

\renewcommand{\indexname}{Index}
\printindex
\end{document}